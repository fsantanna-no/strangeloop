This talk is intended for an audience with an intermediate level of
familiarity with distributed systems, functional programming, and Clojure.

 It has been developed over the past 6 months in response to difficult
analytics problems against large data sets. This talk will cover key design 
techniques, an illustration of the problem that led me to create Onyx, and a 
live demo showing off a solution.


~                                                                            
~                                                                            
~              
\documentclass[11pt,a4paper]{article}
\usepackage[nohead,a4paper,lmargin=1.5cm,rmargin=2.0cm,tmargin=2.0cm,bmargin=2.0cm]{geometry}

\usepackage{verbatim}
\usepackage{xspace}

\newcommand{\CEU}{\textsc{C\'{e}u}\xspace}
\newcommand{\code}[1] {{\small{\texttt{#1}}}}

\title{
    Imperative Reactive Programming with \CEU
}
%\author{Francisco Sant'Anna}

\begin{document}

\maketitle

% wc = 19 19 394

%\abstract{
The origins of reactive programming date back to the early 80's with the 
codevelopment of two styles of synchronous languages:
%
The imperative style of Esterel specifies programs with control flow 
primitives, such as sequences, repetitions, and also parallelism.
%
The dataflow style of Lustre represents programs as graphs of values, in which 
a change to a node updates its dependencies automatically.

In recent years, Functional Reactive Programming modernized the dataflow style 
an became mainstream, producing a number of languages and libraries, such as
Flapjax, Rx (from Microsoft), React (from Facebook), and Elm.
%
In contrast, the imperative style did not follow this trend and is now confined 
to the domain of real-time embedded control systems.
% such as flight control on avionics and anti-collision equipment on automobiles.

We present the programming language \CEU, a contemporary outlook of imperative 
reactivity, aiming to expand the application domain of this family with a new 
abstraction mechanism.
%which supports a rich set of control abstractions and static safety 
%guarantees.
%
\CEU has its roots on Esterel and relies on a similar synchronous and 
deterministic execution model that simplifies the reasoning about concurrency 
aspects.
%
%In particular, the disciplined stepwise synchronous execution of programs 
%allows for concurrent lines to safely share variables.
%
The example below in \CEU prints the "Hello World!" message every second, 
terminating with a key press:

\begin{verbatim}
    par/or do
       every 1s do
          print("Hello World!");
       end
    with
       await KEY;
    end
\end{verbatim}

A \code{par/or} composition spawns concurrent trails (\CEU's lightweight lines 
of execution) that rejoin when any of them terminate (an analogous 
\code{par/and} rejoins when all trails terminate).
%
The first trail continuously prints "Hello World!", while the second awaits for 
a key press to terminate the whole composition.

The \code{await} statement is the most representative of \CEU, capturing the 
imperative and reactive nature of the language at the same time.
%
Awaiting multiple events in parallel compositions releases programmers from the 
infamous \emph{callback hell}, which is often a concern in general purpose 
imperative languages.

%In order to XXX GP
\CEU introduces a new class-based abstraction mechanism that includes an 
execution body with reactive statements.
%
The example below defines a class "Hello" that prints a custom message every 
second.
In the program body, we declare two instances inside a \code{do-end} block that 
limits their scope:
% that goes out of scope after any key press:

\begin{verbatim}
    class Hello with
       var int i;
    do
       every 1s do
          print("I am %d!", this.i);
       end
    end

    do
       var Hello a with
          i = 1;
       end;
       var Hello b with
          i = 2;
       end;
       await KEY;
    end
\end{verbatim}

When the program starts, each instance spawns its body in parallel with the 
\code{do-end} block that will await the key press.
%
On user input, the instances go out of scope with the enclosing block and their 
bodies are automatically terminated.
%
In \CEU, instances have lexical scope just like local primitive variables, 
eliminating the overhead of a garbage collector.
%
An static analysis ensures that references to instances never leak to outer 
scopes.

Even though other languages support local instances (e.g. C++), the presence of 
the \code{await} statement makes possible for blocks in \CEU to outlive 
multiple external reactions while keeping locals alive.
%
Even dynamically allocated instances have lexical scope in \CEU, as illustrated 
below:

\begin{verbatim}
    par/or do
       do
          every 1s do
             spawn Hello with
                i = _rand();
             end;
          end
       end
    with
       await KEY;
    end
\end{verbatim}

We now use the \code{spawn} primitive to create a new instance of \code{Hello} 
every second.
The dynamic instances are bound the enclosing \code{do-end} block.
On user input, the whole \code{par/or} terminates, deallocating all instances 
in the block.
Note how allocated instances are conveniently anonymous, releasing the 
programmer from managing their life cycles.

To evaluate performance, we wrote a stress test program that spawns 10,000 
animated graphical objects, each containing 3 trails (resulting in 30,000 
coexisting trails).
We compared it with a pure event-driven implementation in C and noticed an 
acceptable decrease of 10-20\% in the frame rate for the implementation in 
\CEU.

We believe that the imperative reactive style of \CEU is a realistic 
counterpart to Functional Reactive Programming and complements the realm of 
synchronous programming.
%
The new abstraction mechanism of \CEU makes imperative reactivity suitable for 
applications outside the embedded domain, expanding classical object 
orientation with instances that are reactive to the environment.
%
As an example, we implemented in \CEU an Android game of moderate complexity 
(approximately 5,000 lines of code).

\begin{comment}
programs close to specification

adv: convenience of imperative w/o the burden
also solves the callback hell
- big gap for imperative reactivity

React (vs FRP)
http://www.debjitbiswas.com/genuinely-functional-user-interfaces-and-react/
\end{comment}
%}

% Presentation Outline:
%   - Reactive applications: characteristics, challenges
%   - Reactive models: Declarative/Dataflow/FRP vs Control/Imperative
%   - Overview of Céu
%   - Reactive "Hello World!"
%   - Reactive abstractions in Céu
%   - Demos: ~10 incremental live demonstrations
%
% Previous StrangeLoops?
%   - No, it would my first time.
%
% Speaking Experience:
%   - Talks related to Céu (the topic of this proposal):
%       - 2014 - Arduino Day - Presentation
%                (http://thesynchronousblog.wordpress.com/2014/03/31/ceu-on-arduino-day/)
%       - 2013 - SenSys Conference - Full Paper
%                (http://sensys.acm.org/2013/index.html)
%       - 2013 - REM Workshop (inside OOPSLA) - Full Paper
%                (http://soft.vub.ac.be/REM13/)
%       - 2011 - SenSys Conference - Doctoral Colloquium
%                (http://www.cse.ust.hk/~lingu/SenSys11DC/)
%   - I'm a CS teacher and we've been using Céu in Distributed System classes
%
% Bribes:
%   -
%
%  170   541  6207 controllers.ceu
%   20    51   518 fnts.ceu
%  476  1765 18627 main.ceu
%  575  1828 17676 objs.ceu
%   33    94   918 points.ceu
%   45   120  1351 snds.ceu
%   94   254  2971 texs.ceu
% 1413  4653 48268 total

\end{document}
